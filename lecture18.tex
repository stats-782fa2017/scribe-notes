\documentclass[10pt]{article}
% Include statements
\usepackage{graphicx}
\usepackage{amsfonts,amssymb,amsmath,amsthm}
\usepackage[sort]{natbib}
\usepackage[margin=1in,nohead]{geometry}
\usepackage{multirow,rotating,array}
\usepackage{algorithm,algorithmic}
\usepackage{pdfsync}
\usepackage{hyperref}
\hypersetup{backref,colorlinks=true,citecolor=blue,linkcolor=blue,urlcolor=blue}
\renewcommand{\qedsymbol}{$\blacksquare$}
\setlength{\parindent}{0cm}
\setlength{\parskip}{10pt}


% For sequential numbering
\newcounter{lecnum}
\renewcommand{\thepage}{\thelecnum\ -\ \arabic{page}}
\renewcommand{\thesection}{\thelecnum.\arabic{section}}
\renewcommand{\theequation}{\thelecnum.\arabic{equation}}
\renewcommand{\thefigure}{\thelecnum.\arabic{figure}}
\renewcommand{\thetable}{\thelecnum.\arabic{table}}



% Theorem environments (\autoref compatible)
\usepackage{aliascnt}
\newtheorem{theorem}{Theorem}[lecnum]

\newaliascnt{result}{theorem}
\newtheorem{result}[theorem]{Result}
\aliascntresetthe{result}
\providecommand*{\resultautorefname}{Result}
\newaliascnt{lemma}{theorem}
\newtheorem{lemma}[lemma]{Lemma}
\aliascntresetthe{lemma}
\providecommand*{\lemmaautorefname}{Lemma}
\newaliascnt{prop}{theorem}
\newtheorem{proposition}[prop]{Proposition}
\aliascntresetthe{prop}
\providecommand*{\propautorefname}{Proposition}
\newaliascnt{cor}{theorem}
\newtheorem{corollary}[cor]{Corollary}
\aliascntresetthe{cor}
\providecommand*{\corautorefname}{Corollary}
\newaliascnt{conj}{theorem}
\newtheorem{conjecture}[conj]{Conjecture}
\aliascntresetthe{conj}
\providecommand*{\conjautorefname}{Corollary}
\newaliascnt{def}{theorem}
\newtheorem{definition}[def]{Definition}
\aliascntresetthe{def}
\providecommand*{\defautorefname}{Definition}
\newaliascnt{ex}{theorem}
\newtheorem{example}[ex]{Example}
\aliascntresetthe{ex}
\providecommand*{\exautorefname}{Example}


\newtheorem{assumption}{Assumption}
\renewcommand{\theassumption}{\Alph{assumption}}
\providecommand*{\assumptionautorefname}{Assumption}

\def\algorithmautorefname{Algorithm}
\renewcommand*{\figureautorefname}{Figure}%
\renewcommand*{\tableautorefname}{Table}%
\renewcommand*{\partautorefname}{Part}%
\renewcommand*{\chapterautorefname}{Chapter}%
\renewcommand*{\sectionautorefname}{Section}%
\renewcommand*{\subsectionautorefname}{Section}%
\renewcommand*{\subsubsectionautorefname}{Section}% 


% My Macros
\def\indep{\perp\!\!\!\perp}
\newcommand{\given}{\mbox{ }\vert\mbox{ }}
\newcommand{\X}{\mathcal{X}}
\newcommand{\B}{\mathcal{B}}
\DeclareMathOperator*{\Variance}{Var}
\newcommand{\Var}[1]{\Variance\!\left[#1\right]}
\DeclareMathOperator*{\Covariance}{Cov}
\newcommand{\Cov}[1]{\Covariance\!\left[#1\right]}
\newcommand{\Y}{\mathcal{Y}}
\newcommand{\norm}[1]{\left\lVert #1 \right\rVert}
\newcommand{\email}[1]{\href{mailto:#1}{#1}}
\DeclareMathOperator*{\argmin}{argmin}
\DeclareMathOperator*{\argmax}{argmax}
\newcommand{\indicator}{\mathbbm{1}}
\newcommand{\cdist}{\rightsquigarrow}
\newcommand{\cprob}{\xrightarrow{P}}
\newcommand{\clp}{\xrightarrow{L_p}}
\newcommand{\cas}{\xrightarrow{as}}
\renewcommand{\bar}{\overline}
\renewcommand{\hat}{\widehat}

% Ben's macros:
\newcommand{\F}{\mathcal{F}}
\newcommand{\E}[1]{\mathbb{E}\!\left[#1\right]}
\renewcommand{\P}{\mathbb{P}}
\newcommand{\R}{\mathbb{R}}
\renewcommand{\S}{\mathcal{S}}
\newcommand{\Z}{\mathbb{Z}}
\newcommand{\N}{\mathbb{N}}
\newcommand{\pow}{\raisebox{.15\baselineskip}{\Large\ensuremath{\wp}}}
\renewcommand{\O}{\ensuremath{\mathcal{O}}}
\newcommand{\rad}{\ensuremath{\mathfrak{R}}}
\renewcommand{\F}{\ensuremath{\mathcal{F}}}
\newcommand{\C}{\ensuremath{\mathcal{C}}}
\newcommand{\rhat}{\ensuremath{\hat{R}}}
\newcommand{\st}{\ensuremath{\;\mathrm{s.}\;\mathrm{t.}\;}}
\newcommand{\wrt}{\ensuremath{\;\mathrm{w.}\;\mathrm{r.}\;\mathrm{t.}\;}}
\newcommand{\then}{\ensuremath{\;\Rightarrow\;}}


% To be entered
\setcounter{lecnum}{18}
\newcommand{\lecturer}{Prof.\ McDonald}
\newcommand{\scribe}{Ben Rosenzweig}
\newcommand{\chtitle}{Finite Class Lemma and VC Dimension}
\newcommand{\lecdate}{24 October 2017}


\begin{document}
\rule{6.5in}{1pt}

\textsc{STAT--S 782
        \hfill \thelecnum\ --- \chtitle
        \hfill \lecdate}

\textsc{Lecturer: \lecturer \hfill Scribe: \scribe}
\rule{6.5in}{1pt}

\section{Finite Class Lemma}
Last time, we saw ways to use the concentration of $\Delta_n$ to control $\rad_n(\F)$ (and thus $R(\hat{f})$).
  Today we will discuss other ways to bound Rademacher complexity.

  \begin{lemma}[finite class (v1)]
    If $\mathcal{A}=\{a^{(1)},...,a^{(N)}\}\subset\R^n$ is a finite set with $||a^{(i)}||\leq L \forall i$ and $N\geq2$, then
    \[
    \rad_n(\mathcal{A})\leq \frac{4L\sqrt{\log{N}}}{n}
   \]
  \end{lemma}
  
  \begin{proof}
    By some tricks for sub-Gaussian random variables as in last week's example,
    \begin{align}
      Y_j &= \frac{2}{n} \sum_{i=1}^n \sigma_ia_i^{(j)}\\
    \E{e^{tY_j}} &\leq e^{2L^2t^2/n^2}\\
      \text{ and }\\
        \E{e^{-tY_j}} &\leq e^{2L^2t^2/n^2},\text{  so }\\
   \E{\max_j|Y_j|} &= \E{\max{Y_1,-Y_1,...,Y_N,-Y_N}}\\
    &\leq 2L\sqrt{log2N}/n  (here N\geq2, 2N\leq N^2)\\
    &\leq 4L\sqrt{\log N}/n
    \end{align}

    Thus, with probability $>1-\delta$,
    \[
    R_n(a^{(j)}\leq \rhat_n(a^{(j)}+8L\sqrt{\log N}/n+\sqrt{\frac{\log 1/\delta}{n}}
   \]
  \end{proof}


  \begin{lemma}[finite class (v2)]
      This version of the lemma doesn't use Rademacher complexity.
  \begin{align}
    P(\sup_j|\rhat_n(a^{(j)}) - R_n(a^{(j)})|\geq\epsilon)
  &\leq \sum_{j=1}^N P(|\rhat_n(a^{(j)})-R_n(a^{(j)})|\geq\epsilon)\\
   &\leq 2Ne^{-n\epsilon^2/(2L)^2}\\
  \text{ so w.p. }>1-\delta,\\
  R_n(a^{(j)})&\leq \rhat_n(a^{(j)})+2L\sqrt{\frac{2\log N + \log 1/\delta}{n}}
\end{align}
  \end{lemma}

How can we extend this to infinite classes?
Recall that \F is projected onto the data $z^n:\,\F(z^n) = \{f(z_1),...,f(z_n):f\in\F\}$.
This is the set whose size we want to find.

Suppose $f \mapsto \{0,1\}\,\forall f\in\F$.
Then $\F(z^n)\subseteq\{0,1\}^n$ and thus,
$\forall f\in\F, \, \sqrt{\sum_{i=1}^n|f(z_i)|^2}\leq\sqrt{n}$.

%% |\F(z^n)|\leq 2^n$ (vertex on binary hypercube)

So $\hat{\rad}_n(\F(z^n))\leq4\sqrt{\log|\F(z^n)|/n}$

How can we get a tighter bound?

\section{VC-dimension}
\begin{definition}[Shattering]
  Let \C be a class of subsets of Z.  We say that $S=\{z_1,...,z_n\}\subset\Z$ (finite) 
  is \textbf{shattered} by \C if $\forall S'\subseteq S:\exists C\in \C\st S'=S\cap C.$
  In this case we say C ``picks out'' S'.
  
Equivalently, we can say S is shattered by \C if:
\[
\forall b\in \{0,1\}^n:\exists C\in \C\st (I(z_1\in C),...,I(z_n\in C))=b
\iff \{(I(z_1\in C),...,I(z_n\in C)):C\in\C\} = \{0,1\}^n
  \]
  \end{definition}


Some example classes \C:
\begin{align}
  \C&=\{(-\infty,t]:t\in\R\} \text{, half-open intervals}\\
  \C&=\{(a,b):a\leq b\} \text{, open intervals}\\
  \C&=\{(a,b)\cup(c,d):a\leq b\leq c\leq d\} \text{, unions of open intervals}\\
\C&= \text{ all discs in }\R^d \\
\C&= \text{all axis parallel rectangles in }\R^d\\
\C&= \{x:\beta^Tx\geq0\} \text{, all half spaces}\\
\C&= \text{all convex sets}
    \end{align}

\begin{example}
Let $\C=\{(a,b):a\leq b\}, S=\{1,2,3\}$.
Which $S'\subset S$ can \C pick out?
\C picks out all $S'\in\pow(S)$ EXCEPT $\{1,3\}$.  So S is not shattered by \C.

More generally, $\C=\{(a,b):a\leq b\}$ can shatter any set of size two ($S=\{s,t\}$ where $ s\neq t$),
but only trivial sets of size 3.
\end{example}    
  
\begin{definition}[Vapnik-Chervonenkis (VC) dimension of \C]
  The \textbf{VC-dimension} is 
  \[
  V(\C) := \max\{n\in\N:\exists S\subset\Z\st |S|=n \text{ and S is shattered by \C}\}.
  \]
  If $V(\C)<\infty$ we say \C is a \textbf{VC-class}.
 \end{definition}

The hard part is always proving that no set of size $> n$ can be shattered.

\begin{definition}[the $n^{th}$ shatter coefficient]
$\mathcal{S}_n(\C):=\sup_{S\subset\Z,|S|=n}|\{S\cap C:C\in\C\}|$
\end{definition}
The shattering coefficient gives us another way to define VC-dimension: $V(\C) := \max\{n\in\N:\S_n(\C)=2^n\}$.
VC is always well-defined, since $\S_n<2^n \then \S_m<2^m\,\forall m>n$.

\begin{example}
  For $\C=\{(a,b):a\leq b\},\,\S_2(\C)=4,\,\S_3(\C)=7$.
\end{example}    

\begin{example}
  VC-dim for binary functions:
  $\forall f:\Z\to\{0,1\},  let C_f=\{z:f(z)=1\}$.
  
$\forall C\subseteq\Z$, let $f_C(x)=I(x\in C)$.
\end{example}

\begin{definition}
Let \F be a class of functions $f:\Z\to\{0,1\}$.  We say S is shattered by \F if S is shattered by $C_\F = \{I(f=1):f\in\F\}$,
where $I(f=1)$ is the indicator of $C_f.$  In addition, let
\begin{align}
\S_n(\F) &:=\S_n(\C_\F)\\
V(\F)&:=V(\C_\F)
\end{align}
\end{definition}

\begin{example}[semi-infinite intervals]
For $\C = \{(-\infty,t):t\in\R\}$, $V(\C) = 1$. Clearly, \C can shatter some 1-point set: take $S=\{0\}$.  Then $(-\infty,-1]\cap S = \emptyset$ and $ (-\infty,-1]\cap S = S.$
    So we know $V(\C)\geq 1$.
    
To show there is no 2-point set,
let $S=\{a,b\}, a<b.$
Obviously, $\not\exists t\st(-\infty,t]=\{b\}.$
\end{example}

\begin{example}[open intervals]
  Let $\C=\{(a,b):a\leq b\}$.
\begin{enumerate}
  \item $S=\{s,t\}\, s<t$.  choose $a_1<a_2<s<a_3<t<a_4$.
Then \begin{align} (a_1,a_2)\cap S&=\emptyset\\
(a_2,a_3)\cap S&=\{s\}\\
(a_1,a_4)\cap S&=\{t\}\\
\end{align}
\item For any $S=\{s,t,u\} (s<t<u)$ and $a_1<a_2$,\\
 $ \{s,u\}\subset(a_1,a_2) \then \{t\}\subset(a_1,a_2)$
\end{enumerate}
\end{example}

%% algorithmic stability, bound on rad comp (which you can't use directly for real-valued case unless loss is bdd)
%% params of k-SVM infinite, but # support vecs finite = VC-dim 

\end{document}
