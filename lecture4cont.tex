\documentclass[10pt]{article}
% Include statements
\usepackage{graphicx}
\usepackage{amsfonts,amssymb,amsmath,amsthm}
\usepackage[sort]{natbib}
\usepackage[margin=1in,nohead]{geometry}
\usepackage{multirow,rotating,array}
\usepackage{algorithm,algorithmic}
\usepackage{pdfsync}
\usepackage{hyperref}
\usepackage{braket}
\hypersetup{backref,colorlinks=true,citecolor=blue,linkcolor=blue,urlcolor=blue}
\renewcommand{\qedsymbol}{$\blacksquare$}
\setlength{\parindent}{0cm}
\setlength{\parskip}{10pt}


% For sequential numbering
\newcounter{lecnum}
\renewcommand{\thepage}{\thelecnum\ -\ \arabic{page}}
\renewcommand{\thesection}{\thelecnum.\arabic{section}}
\renewcommand{\theequation}{\thelecnum.\arabic{equation}}
\renewcommand{\thefigure}{\thelecnum.\arabic{figure}}
\renewcommand{\thetable}{\thelecnum.\arabic{table}}



% Theorem environments (\autoref compatible)
\usepackage{aliascnt}
\newtheorem{theorem}{Theorem}[lecnum]

\newaliascnt{result}{theorem}
\newtheorem{result}[theorem]{Result}
\aliascntresetthe{result}
\providecommand*{\resultautorefname}{Result}
\newaliascnt{lemma}{theorem}
\newtheorem{lemma}[lemma]{Lemma}
\aliascntresetthe{lemma}
\providecommand*{\lemmaautorefname}{Lemma}
\newaliascnt{prop}{theorem}
\newtheorem{proposition}[prop]{Proposition}
\aliascntresetthe{prop}
\providecommand*{\propautorefname}{Proposition}
\newaliascnt{cor}{theorem}
\newtheorem{corollary}[cor]{Corollary}
\aliascntresetthe{cor}
\providecommand*{\corautorefname}{Corollary}
\newaliascnt{conj}{theorem}
\newtheorem{conjecture}[conj]{Conjecture}
\aliascntresetthe{conj}
\providecommand*{\conjautorefname}{Corollary}
\newaliascnt{def}{theorem}
\newtheorem{definition}[def]{Definition}
\aliascntresetthe{def}
\providecommand*{\defautorefname}{Definition}
\newaliascnt{ex}{theorem}
\newtheorem{example}[ex]{Example}
\aliascntresetthe{ex}
\providecommand*{\exautorefname}{Example}


\newtheorem{assumption}{Assumption}
\renewcommand{\theassumption}{\Alph{assumption}}
\providecommand*{\assumptionautorefname}{Assumption}

\def\algorithmautorefname{Algorithm}
\renewcommand*{\figureautorefname}{Figure}%
\renewcommand*{\tableautorefname}{Table}%
\renewcommand*{\partautorefname}{Part}%
\renewcommand*{\chapterautorefname}{Chapter}%
\renewcommand*{\sectionautorefname}{Section}%
\renewcommand*{\subsectionautorefname}{Section}%
\renewcommand*{\subsubsectionautorefname}{Section}% 


% My Macros
\def\indep{\perp\!\!\!\perp}
\newcommand{\given}{\mbox{ }\vert\mbox{ }}
\newcommand{\F}{\mathcal{F}}
\newcommand{\Expect}[1]{\mathbb{E}\!\left[#1\right]}
\renewcommand{\P}{\mathbb{P}}
\newcommand{\R}{\mathbb{R}}
\newcommand{\X}{\mathcal{X}}
\newcommand{\B}{\mathcal{B}}
\DeclareMathOperator*{\Variance}{Var}
\newcommand{\Var}[1]{\Variance\!\left[#1\right]}
\DeclareMathOperator*{\Covariance}{Cov}
\newcommand{\Cov}[1]{\Covariance\!\left[#1\right]}
\newcommand{\Y}{\mathcal{Y}}
\newcommand{\norm}[1]{\left\lVert #1 \right\rVert}
\newcommand{\email}[1]{\href{mailto:#1}{#1}}
\DeclareMathOperator*{\argmin}{argmin}
\DeclareMathOperator*{\argmax}{argmax}
\newcommand{\indicator}{\mathbbm{1}}
\newcommand{\cdist}{\rightsquigarrow}
\newcommand{\cprob}{\xrightarrow{P}}
\newcommand{\clp}{\xrightarrow{L_p}}
\newcommand{\cas}{\xrightarrow{as}}
\renewcommand{\bar}{\overline}
\renewcommand{\hat}{\widehat}


% Your new macros

\newcommand{\Rn}{\R^n}
\newcommand{\sumi}[2]{\sum\limits_{#1=1}^{#2}}
\newcommand{\TODO}[1]{\textbf{\color{red} TODO: #1}}
\newcommand{\trans}[1]{{#1}^\mathsf{T}}
\newcommand{\Sym}{\mathbb{S}^n}
\DeclareMathOperator*{\dom}{dom}
\newcommand{\minl}{\min\limits}


%%%%%%%%%%%%%%%%%%%%%%%%%%%%%%%%
% To be entered
\setcounter{lecnum}{4}
\newcommand{\lecturer}{Prof.\ McDonald}
\newcommand{\scribe}{Inhak Hwang}
\newcommand{\chtitle}{Optimization basics (cont.)}
\newcommand{\lecdate}{5 September 2017}


\begin{document}
\rule{6.5in}{1pt}

\textsc{STAT--S 782
        \hfill \thelecnum\ --- \chtitle
        \hfill \lecdate}

\textsc{Lecturer: \lecturer \hfill Scribe: \scribe}
\rule{6.5in}{1pt}



\section{First Order Condition}
\label{sec:terminology}

\begin{itemize}
  \item convex problem with differentialbe $`f'$
  \item a feasible $x$ is optimal iff $\nabla f(x)^{T}( x - y) \ge 0$, $\forall$ feasible $y$
  \item if unconstrained, the condition reduces to $\nabla f(x)  = 0 $
\end{itemize}


\begin{example}
  $\minl_x \frac{1}{2}x^{T}Q x + b^{T}x + c $ , $Q \succeq 0$ \\
      $\qquad$FOC: $\nabla f(x) = Q^{T}x + b = 0$  \\
	$\qquad$ if $Q \succ 0 $ $\rightarrow$ $x^{*} = -Q^{-1}b$ \\
	$\qquad$ if $Q$ singular, $b \in Col[Q] \rightarrow$ no solution \\
     $\qquad$ if $Q$ singular, $b \notin Col[Q] \rightarrow$ $x^{*} = -Q^{*}b + z $ with $z \in null[Q]$
\end{example}

\begin{example}
 Projection ont convex $C$ : $\min\limits_{x} \|a-x\|_{2}^{2}$ s.t. $x\in C$\\
$\qquad$ FOC : $\nabla f(x)^{T}(y-x) = (x - a)^{T}(y - x) \ge 0$, $\forall y\in C$ $\Leftrightarrow$ $a-x \in N_{2}(x)$
\end{example}


\section{Useful Operations}
\subsection{Partial Optimizations}
$h(x) = \min\limits_{y}f(x,y) $ is convex if $f$ is convex, and $C$ is convex.
\begin{example}
$\min\limits_{x_{1},x_{2}} f(x_{1},x_{2})$  $s.t. g_{1}(x_{1}) \le 0, g_{2}(x_{2}) \le 0$  $\Leftrightarrow$ $ \min\limits_{x_{1}}h(x_{1})$ s.t. $g_{1}(x_{1}) \le 0$
\end{example}

\subsection{Transformations}
We can use a monotone increasing transformation $h: \mathbb{R} \rightarrow \mathbb{R}$ to change our problem: \\
$\quad \min\limits_{x\in C} f(x) \Rightarrow \min\limits_{x\in C} h(f(x)) $ \\
We can use a change of variable transformation $\phi : \mathbb{R}^{n} \Rightarrow \mathbb{R}^{m}$ :  \\
$\quad \min\limits_{x \in C} f(x) \Leftrightarrow \min\limits_{\phi(y) \in C} f(\phi(y))$

\begin{example}
Geometric Program \\
$\min\limits_{x} f(x) = \sum\limits_{k=1}^{p}\gamma_{k}x_{1}^{a_{k_1}}x_{2}^{a_{k_{2}}} ..x_{n}^{a_{k_{n}}} $ (posynomial) \\
 $C$ : involved inequalities are in some form and equalities are affine.  \\
We can change above non-convex problem to the following convex problem by letting $y_{i} = log x_{i}$ : \\
$\min\limits_{y} \log(\sum\limits_{k=1}^{p}\exp^{a_{ik}^{T}y + b_{ik}}) $ s.t. $\log(\sum_{k=1}^{m} \exp^{a_{ik}^{T}y + b_{ik}}) \le 0$ and $c_{j}^{T}y + d_{j} = 0, j = 1, ... , r $
\end{example}

\subsection{Eliminate equality constraints}
$\min\limits_{x}f(x)$ s.t. $g_{i}(x) \le  0, Ax = b$ \\
$x$ feasible = $My + x_{0}$ s.t. $Ax_{0} = b$ $col[M]= null[A]$ \\
So we can get an optimization problem with only inequality constraints :  \\
$\min\limits_{y} f[My+ x_{0}]$ 	s.t. $ g_{i}(My+x_{i}) = 0$

\subsection {Slack Variables}
Consider: \\
$\min\limits_{x,s}f(x) $ s.t. $s \ge 0, g_{i}(x) + s_{i} = 0, Ax= b$
This problem is not convex unless $g_{i}$ is affine. We can relx nonaffine constraints to get : \\
$\min\limits_{x \in C} f(x) \Rightarrow \min\limits_{x \in \tilde{C} } f(x)$,   $C \subset \tilde{C}$ \\
In this case optimum of new problem is smaller or equal to the optimum of the original problem.

\section{Standard Problems}
\subsection{LP (Linear Programs) }
$\qquad \min_{x} c^{T}x $ with affine inequlaity and affine eqaulity

\begin{example}
Basis Pursuit \\
$\qquad \min\limits_{\beta} \| \beta_{0} \|$  s.t. $X \beta = y$  \\
$\quad$Above problem can be relaxed to : \\
$\qquad \min\limits_{\beta} \| \beta \|_{1}$ s.t. $X \beta = y$. \\
$\quad$ This relaxation can be reformulated to a LP problem: \\
$\qquad \min\limits_{\beta, z} 1^{T}z$ s.t. $z \ge \beta, z \ge -\beta, X\beta = y$
\end{example}

\begin{example}
Dantzig selector \\
$\qquad \min\limits_{\beta} \| \beta \|_{1} $ s.t. $\|x^{T}(y - X\beta)\|_{\infty} \le \lambda$
\end{example}

\subsection{ QP (Quadratic Programming) }
\begin{example}
Lasso, ridge regression, OLS, Portfolio Optimization 
\end{example}

\subsection {SDP (Semi-Definite Programming)}
$\qquad \min\limits_{X \in S_{n}} tr(C^{T}X)$ s.t. $tr(A_{i}^{T}X) = b_{i}$, $X \succeq 0$ \\
$\qquad$ We define $X \bullet A = tr(X^{T}A) $ from now on.

\subsection {Conic Program}
$\qquad \min\limits_{x}c^{T}x$ \\s.t. $Ax = b$, $D(x) + d \in K$, $K$ a closed convex cone.

The following relationship holds between above programs: \\
$CP(Conic Programming) \subset QP \subset SOCP \subset SDP \subset CP(Convex Programming)$ 

\section{Lower bound in LP}
Want to find $B \le \min\limits_{x}f(x)$ 

\begin{example}
easy example \\
$\min\limits_{x,y} x+y $ s.t. $x+y \ge 2, x \ge 0, y \ge 0$
\end{example}


\begin{example}
medium example \\
$\min\limits_{x,y} x + 3y $ s.t. $x + y \ge 2, x \ge 0, y \ge 0$ \\
We transform constraints : \\
know $x + y \ge 2, y \ge 0 \Rightarrow$ $ x + y \ge 2, 2y \ge 0$ $\Rightarrow x + 3y \ge 2$
\end{example}

\begin{example} general LP \\
$\min\limits_{x,y} px + qy$ s.t. $ x + y \ge 2, x ,y \ge 0 $ \\
We transform constraints: \\
$ax + ay \ge 2a, bx \ge 0, cy \ge 0, a \ge 0, b \ge 0  c \ge 0$ \\
Add $(a+b)x + (a +c) y \ge 2a $ \\ 
Let $(a+b) = p, (a+c) =q \Rightarrow b = 2a$ \\
To answer the question ``What is best lower bound?", Solve: \\
$\qquad \max\limits_{a,b,c} 2a $\\ s.t. $ a + b = p, a + c =q , a, b, c \ge 0$ \\
The above LP is "dual" of the original LP : \\
$\qquad \min\limits_{x,y} px + qy$ s.t. $ x+y \ge 2, x, y, \ge 0$ \\
We see that number of Dual variables = number of Primal constraints 
\end{example}

\begin{example} Another general LP example \\
$\qquad \min\limits_{x,y}(px + qy) $ 	s.t. $x \ge 0, y \le 1, 3x + y = 2$ \\
$\qquad$We transform constraints: \\
$\qquad a \ge 0, b \ge 0, -by \ge -b, 3cx + cy = 2c$ \\
$\qquad$ Now combine: \\
$\qquad (a + 3c)x + (-b +c )y \ge 2c - b$ \\ 
$\qquad$ By letting p = (a+3c) and q = (-b + c) We get the following Dual: \\
$\qquad \max\limits_{a,b,c} 2c - b$ \\
$\qquad$ s.t. $ a +3c = p, c- b =a , a, b \ge 0$
\end{example}

For General LP: \\
$\qquad \min\limits_{x}c^{T}x $ \\
$\qquad$ s.t. $Ax = b, Gx \le h$ \\
$ \qquad \Rightarrow$ \\
$\qquad \max_{u,v} -b^{T}w -b^{T}v $ \\
$\qquad$ s.t. $-A^{T}u - h^{T}v = c, v\ge 0$



%\bibliographystyle{scribebibsty}
%\bibliography{s782references}

\end{document}
 