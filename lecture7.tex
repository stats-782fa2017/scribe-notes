\documentclass[10pt]{article}
% Include statements
\usepackage{graphicx}
\usepackage{amsfonts,amssymb,amsmath,amsthm}
\usepackage[sort]{natbib}
\usepackage[margin=1in,nohead]{geometry}
\usepackage{multirow,rotating,array}
\usepackage{algorithm,algorithmic}
\usepackage{pdfsync}
\usepackage{hyperref}
\hypersetup{backref,colorlinks=true,citecolor=blue,linkcolor=blue,urlcolor=blue}
\renewcommand{\qedsymbol}{$\blacksquare$}
\setlength{\parindent}{0cm}
\setlength{\parskip}{10pt}


% For sequential numbering
\newcounter{lecnum}
\renewcommand{\thepage}{\thelecnum\ -\ \arabic{page}}
\renewcommand{\thesection}{\thelecnum.\arabic{section}}
\renewcommand{\theequation}{\thelecnum.\arabic{equation}}
\renewcommand{\thefigure}{\thelecnum.\arabic{figure}}
\renewcommand{\thetable}{\thelecnum.\arabic{table}}



% Theorem environments (\autoref compatible)
\usepackage{aliascnt}
\newtheorem{theorem}{Theorem}[lecnum]

\newaliascnt{result}{theorem}
\newtheorem{result}[theorem]{Result}
\aliascntresetthe{result}
\providecommand*{\resultautorefname}{Result}
\newaliascnt{lemma}{theorem}
\newtheorem{lemma}[lemma]{Lemma}
\aliascntresetthe{lemma}
\providecommand*{\lemmaautorefname}{Lemma}
\newaliascnt{prop}{theorem}
\newtheorem{proposition}[prop]{Proposition}
\aliascntresetthe{prop}
\providecommand*{\propautorefname}{Proposition}
\newaliascnt{cor}{theorem}
\newtheorem{corollary}[cor]{Corollary}
\aliascntresetthe{cor}
\providecommand*{\corautorefname}{Corollary}
\newaliascnt{conj}{theorem}
\newtheorem{conjecture}[conj]{Conjecture}
\aliascntresetthe{conj}
\providecommand*{\conjautorefname}{Corollary}
\newaliascnt{def}{theorem}
\newtheorem{definition}[def]{Definition}
\aliascntresetthe{def}
\providecommand*{\defautorefname}{Definition}
\newaliascnt{ex}{theorem}
\newtheorem{example}[ex]{Example}
\aliascntresetthe{ex}
\providecommand*{\exautorefname}{Example}


\newtheorem{assumption}{Assumption}
\renewcommand{\theassumption}{\Alph{assumption}}
\providecommand*{\assumptionautorefname}{Assumption}

\def\algorithmautorefname{Algorithm}
\renewcommand*{\figureautorefname}{Figure}%
\renewcommand*{\tableautorefname}{Table}%
\renewcommand*{\partautorefname}{Part}%
\renewcommand*{\chapterautorefname}{Chapter}%
\renewcommand*{\sectionautorefname}{Section}%
\renewcommand*{\subsectionautorefname}{Section}%
\renewcommand*{\subsubsectionautorefname}{Section}% 


% My Macros
\def\indep{\perp\!\!\!\perp}
\newcommand{\given}{\mbox{ }\vert\mbox{ }}
\newcommand{\F}{\mathcal{F}}
\newcommand{\Expect}[1]{\mathbb{E}\!\left[#1\right]}
\renewcommand{\P}{\mathbb{P}}
\newcommand{\R}{\mathbb{R}}
\newcommand{\X}{\mathcal{X}}
\newcommand{\B}{\mathcal{B}}
\DeclareMathOperator*{\Variance}{Var}
\newcommand{\Var}[1]{\Variance\!\left[#1\right]}
\DeclareMathOperator*{\Covariance}{Cov}
\newcommand{\Cov}[1]{\Covariance\!\left[#1\right]}
\newcommand{\Y}{\mathcal{Y}}
\newcommand{\norm}[1]{\left\lVert #1 \right\rVert}
\newcommand{\email}[1]{\href{mailto:#1}{#1}}
\DeclareMathOperator*{\argmin}{argmin}
\DeclareMathOperator*{\argmax}{argmax}
\newcommand{\indicator}{\mathbbm{1}}
\newcommand{\cdist}{\rightsquigarrow}
\newcommand{\cprob}{\xrightarrow{P}}
\newcommand{\clp}{\xrightarrow{L_p}}
\newcommand{\cas}{\xrightarrow{as}}
\renewcommand{\bar}{\overline}
\renewcommand{\hat}{\widehat}


% Your new macros



% To be entered
\setcounter{lecnum}{7}
\newcommand{\lecturer}{Prof.\ McDonald}
\newcommand{\scribe}{Yisu Peng}
\newcommand{\chtitle}{Duals and Conjugates}
\newcommand{\lecdate}{14 September 2017}


\begin{document}
\rule{6.5in}{1pt}

\textsc{STAT--S 782
        \hfill \thelecnum\ --- \chtitle
        \hfill \lecdate}

\textsc{Lecturer: \lecturer \hfill Scribe: \scribe}
\rule{6.5in}{1pt}

Given problem,
\begin{equation}
\left\{
\begin{array}{lll}
&\min\limits_x f(x)\\
\text{s.t. } &h_i(x) \le 0\\
&l_j(x) = 0
\end{array}
\right.
\end{equation}
The KKT conditions are
\begin{enumerate}
\item $0 \in \partial(f(x) + u^Th(x) + v^T l(x))$
\item $u_i h_i = 0, \forall i$
\item $h_i(x) \le 0, l_i(x) = 0$
\item $v_i \ge 0, \forall i$
\end{enumerate}

\begin{itemize}
\item Necessary (conditions) for optimality, under strong duality.
\item Always sufficient for optimality.
\item Duality gap\\
for $x (u,v)$ feasible, $f(x) - f(x^*) \le f(x) - g(u,v)$
\item Strong duality\\
for some feasible $(u^*, v^*)$, any primal solution $x^*$ minimizes $L(x,u^*,v^*)$
\end{itemize}


\section{Conjugate Function}

Given $f: \mathbb{R}^n \rightarrow \mathbb{R}$, $f^*: \mathbb{R}^n \rightarrow \mathbb{R}$
\begin{equation}
f^*(y) = \max\limits_x y^T x - f(x)
\end{equation}


\textbf{Note:} The conjugate function is always convex.

``Largest gap between the line $y^T x$ and $f(x)$".

If $f$ is differentiable, then this is called ``Legendre transform'' of $f$.

\subsection*{Properties}
\begin{itemize}
\item Fenchel's inequality $\forall x, y$
\begin{equation}
f(x) + f^*(y) \ge x^T y
\end{equation}
\item $ \Longrightarrow f^{**} \le f$
\item if $f$ is closed, convex, then $f^{**} = f$
\item if $f$ is closed, convex, then $\forall x, y$
\begin{align}
x \in \partial f^*(y) & \Longleftrightarrow y \in \partial f(x)\\
	& \Longleftrightarrow f(x) + f^*(y) = x^T y
\end{align}
\item If $f(u,v) = f_1(u) + f_2(v)$, then
\begin{equation}
f^*(w,z) = f^*_1(w) + f^*_2(z)
\end{equation}

\item If $f(x) = a g(x) + b$, then
\begin{equation}
 f^*(y) = a g^*(\frac{y}{a}) - b
\end{equation}
\item If $A\in \R^{n\times n}$, non-singular, $f(x) = g(Ax+b)$, then
\begin{equation}
f^*(y) = g^*(A^{-T} y) - b^T A^{-T} y
\end{equation}
\end{itemize}

\subsection*{Examples}
\begin{enumerate}
\item $f(x) = x^T Q x$,  $Q \succ 0\\$
\begin{align*}
\max\limits_x y^T x - \frac{1}{2} x^T Q x\\
\end{align*}
Take partial derivative towards $x$ and set to 0,
\begin{align*}
y - Qx = 0 \Longrightarrow Q^{-1} y = x^*\\
f^*(y) = \frac{1}{2} y^T Q^{-1} y
\end{align*}
\item If $f(x) = I_C(x) = \left\{
\begin{array}{ll}
0 &x\in C,\\
\infty &\text{elsewhere.}
\end{array}
\right.$
\begin{align*}
f^*(y) = \max\limits_{x\in C} y^T x &&\text{support function}
\end{align*}
\item Dual norms\\
Given a norm $||\cdot||$,
\begin{align*}
|y^T x| \le \|x\| \|y\|_*
\end{align*}
For Euclidean norm, the conjugate is still the Euclidean norm, and the inequality is called Cauchy-Schwarz inequality.

For $\ell_p$-norm,
\begin{align*}
(||x||_p)_* = ||y||_q\\
\frac{1}{q} + \frac{1}{p} = 1
\end{align*}
Trace norm
\begin{align*}
(\|X\|_{tr})_* = \| X \|_{op} = \sigma_1(x)\\
\|X\|_{**} = \|X\|
\end{align*}
Conjugate $f(x) = \|x\|$
\begin{equation*}
f^*(y) = I_{\{z:\|z\|_*\le 1\}} (y)
\end{equation*}
\item Squared norms $f(x) = \frac{1}{2} \| x \| ^2$,
\begin{equation*}
f^*(y) = \frac{1}{2} \| y \|_*^2
\end{equation*}
\item Affine function
$f(x) = a^Tx + b$
\begin{equation*}
f^*(x) = \max\limits_x y^T x - a^T x - b
\end{equation*}
bounded if and only if $y = a$.
\begin{equation*}
f^*(y) = \left\{
\begin{array}{ll}
-b &  y = a\\
\infty & \text{else}
\end{array}
\right.
\end{equation*}
\item $f(x) = -\log x \qquad \text{dom}(f): x > 0$
\begin{equation*}
f^*(y) = \max\limits_x y^T x + \log x
\end{equation*}
unbounded if $y \ge 0$\\
$x^* = -\frac{1}{y}$
\begin{equation*}
f^*(y) = \left\{
\begin{array}{ll}
-\log(-y) - 1 & y<0\\
\infty	& \text{else}
\end{array}\right.
\end{equation*}
Why?
\begin{equation*}
-f^*(u) = \min\limits_x f(x) - u^T x
\end{equation*}
Try $\min\limits_x f(x) + g(x)$
\begin{align*}
\Longleftrightarrow \min\limits_{x, z} f(x) + g(z)\\
\text{s.t. } x = z
\end{align*}
\begin{align*}
g(u) &= \min\limits_{x, z} f(x) + g(z) + u^T(z-x)\\
&= \min\limits_x \{f(x) - u^T x\} + \min_z\{g(z) - (-u)^T z\}\\
&= -\max\limits_x\{u^Tx - f(x)\} - \max\limits_z\{(-u)^T z - g(z)\}\\
&= -f^*(u) - g^*(-u)
\end{align*}
\item Dual of $\min\limits_x f(x) + \|x\|$ is
\begin{equation*}
\max\limits_u - f^*(u) - I_{\{z:\|z\|_*\le 1\}}(u)
\end{equation*}
\textbf{Ex}. Lasso
\begin{align*}
\min\limits_\beta \frac{1}{2} \|y - X\beta \|_2^2 + \lambda \|\beta\|_1\\
\Longleftrightarrow \min\limits_{\beta, z}\frac{1}{2} \|y - z\|_2^2 + \lambda\|\beta\|_1\\
\text{s.t. } z = X\beta
\end{align*}
Dual of $\frac{1}{2}\|v\|_2^2$ is $\frac{1}{2} \|v\|_2^2$
\begin{equation*}
g(u) = - \frac{1}{2}\|u\|_2^2 + y^T u - I_{\{v:\|v\|\le 1\}} (\frac{X^T u}{\lambda})
\end{equation*}
\end{enumerate}


\bibliographystyle{scribebibsty}
\bibliography{s782references}

\end{document}
